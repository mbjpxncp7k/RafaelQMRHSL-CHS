\chapter{导言}

本章将概述本论文的内容。这些内容主要是关于\dirac 右矢、\lippmann-\schwinger 右矢、\gamow 向量的性质。

% quote

本文是关于\dirac 右矢、\lippmann-\schwinger 右矢、\gamow 向量在\rhs 语言中的表述。
\dirac 右矢是关联于一个可观测量的谱点的态向量。
\lippmann-\schwinger 右矢是与散射理论有关的\hamiltonian 的本征右矢。
它们对应于\longRadicalTranslation{等能}{monoenergetic}“入”散射态和“出”散射态。
\gamow 向量是表示共振的态向量的右矢。
我们的主要目标是展现:\rhs 是对刻画这些右矢而言最合适的框架。
我们将用例子来说明这一点,而不是完全停留于抽象的讨论。主要用到的两个例子是谐振子和方势垒。

本文未讨论任何实验数据,而是关注于解释和理解这些数据的方法、思想和原则。
我们将用配备各种边界条件的\schrodinger 方程来作为提供数据的模型。
\schrodinger 方程上的各种边界条件将给出\dirac 右矢、\lippmann-\schwinger 右矢、\gamow 向量。
这样的模型涉及了一种理想化,但这可能仍是理解这些态向量为何物的最佳方式。

须注意,\RHS\emph{不是}对量子力学的一种阐释,而是一种表述如\dirac 右矢、\lippmann-\schwinger 右矢、\gamow 向量这样的启发性的物理概念的更自然、精确、逻辑化的语言。

\section{\rhs 简史}

在1920年代末,\dirac 引入了量子力学的一种新数学模型,它是基于无穷维复内积向量空间上线性算子所构的一种\longRadicalTranslation{唯一的光滑而精致}{uniquely smooth and elegant}的抽象代数。%cite
\dirac 的\terminology{左矢}、\terminology{右矢}(源于内积的括号记法)的抽象代数模型在之后的年头里已证明是有着巨大的启发价值——尤其是在处理具有连续谱的\hamilton 算子时。
然而,在寻找一种可用于实际数值计算的线性代数时则出现了严重的困难。

\introduceAbbreviation{\hs}{HS}是为量子力学提出的第一种数学\shortRadicalTranslation{形式化}{idealization}模型
然而,如\neumann 在他的书中解释的那样,\HS 理论和\dirac 体系是两回事。尽管有在\hs 中实现\dirac 模型的尝试,仍有不少严重问题使得这体系无法给出左矢、右矢或\deltafunc ,或是无法给\dirac 基向量展开以一个数学含义——而这种展开在涉及连续谱的量子力学的物理表述中都是及其重要的。
当然,\dirac 在他的文字报告中表示:\quoteText{我们目前所用的左矢和右矢构成了一个比\hs 更一般的空间。}。%cite

在1940年代末,%cite
为\deltafunc 给出了精确的含义,它定义为一种测试函数空间上的线性泛函。
这导向了泛函分析的一个新分支的发展:分布理论。%cite

大约是与此同时,\neumann 发表了\hs 的由一个自伴算子诱导的直积分分解理论(也对更一般的情况成立)。%cite
这种谱理论更接近于经典\fourier 分析,它代表了对更早的\neumann 谱理论的一项改进。%cite

%cite
则总是认为\neumann 谱理论并非无穷维向量空间上线性算子理论的完整故事。
获益于分布理论的发展,他与其学派引入了\introduceAbbreviation{\rhs}{RHS}。
从这\rhs 和\neumann 的直积分分解出发,他们得以证明所谓核型谱定理(也称为\gelfand-\maurin 定理)。%cite
该定理为算子的谱性质提供了更为彻底的信息,并在相同立足点上处理了连续谱与离散谱。

作为\dirac 体系的一面,可观测量代数中元素的连续性在1960年代初得到了讨论。%cite
若可观测量代数中的两个算子满足\canonical(\heisenberg)对易关系,则它们中至少有一个不能在\hs 拓扑下是连续的(即,有界的)。
%cite
证明了\hs 中有这样一些子定义域,其上可配备使得那些算子连续的拓扑。而这些子定义域中最大的就是\schwartz 空间。

在1960年代末,一些物理学家独立地意识到\RHS 为\dirac 体系的所有方面都提供了一个严格的数学表述。%cite
特别地,核型谱定理在数学理论中重新表述了\dirac 基向量展开和\dirac 左、右矢。
之后,其他作者也得到了相同结论。%cite
如今,\RHS 已成为教科书中的内容。%cite

在过去一些年中,\RHS 作为散射与衰变理论的自然的数学语言而出现。%cite
\RHS 也被证明在其他理论物理领域中很有用处,如混沌映射的广义谱分解的构造。%cite
实际上,为以一种\longRadicalTranslation{自恰}{consistent}的方式来处理散射与衰变,\RHS 似乎是最为人所知的语言。
这就是我们为何在此使用它。

\schrodinger 方程是统御量子系统行为的动力学方程。
于是任何试图证明\RHS 包含量子力学所需的数学方法的尝试,都应展现\RHS 是\schrodinger 方程\shortRadicalTranslation{之解}{solutions of}的自然框架。
本论文的目的在于得到作为不同边界条件下\schrodinger 方程的解的\dirac、\lippmann-\schwinger、\gamow 右矢,
并展示这些解是落在\RHS 中而非仅仅是在\HS 中。%cite

最终,本文的这些结论将允许我们描绘出一个非常重要的结论:\RHS 是处理散射与衰变的自然语言。

\section{谐振子}

若一可观测量的谱是离散的,则\hs 的数学方法对于量子力学而言已是充分的。
然而,若一可观测量的谱有连续的部分,\hs 的数学方法则还不够,需要这些方法的一种扩展。

物理学家使用\dirac 的左右矢体系来处理连续谱。
四种这种体系的最重要特性是:
\begin{enumerate}
	\item 对于\(A\)中谱的任一元素\(\lambda\),都有一对应的右矢\(\ket\lambda\),其是\(A\)的关于本征值\(\lambda\)的本征矢:
	\[\label{eq:dirac_formalism_feature_1}
		A\ket\lambda=\lambda\ket\lambda.\]
	\item 一个波函数\(\varphi\)可用这些本征右矢展开\footnote{下式被称为\dirac 基向量展开}:
	\[\label{eq:dirac_formalism_feature_2}
		\varphi=\int_{\operatorname{Spectrum}(A)}\dd\lambda\ket\lambda\braket{\lambda|\varphi}.\]
	\item 本征右矢按下面的规则归一化:
	\[\braket{\lambda|\lambda'}=\delta(\lambda-\lambda'),\]
	其中\(\delta(\lambda-\lambda')\)是\deltafunc 。
	\item 所有代数运算,如两个可观测量\(A,B\)间的对易子
	\[\label{eq:dirac_formalism_feature_4}
		\cmtt{A}{B}\defeq AB-BA.\]
	都是良定义的。
\end{enumerate}
	
在量子力学中,可观测量被假定是定义于\hs\(\cH\)上的自伴线性算子。
若关联于一个可观测量的算子\(A\)是无界的(这是量子力学中最常见的情况),\(A\)则仅定义在一个子域\(\cD(A)\)上,在其上\(A\)得以是自伴的。
在这种情况下,\hs 方法并不足以使\crefrange{eq:dirac_formalism_feature_1}{eq:dirac_formalism_feature_4}有意义。
而\RHS 体系则提供了使它们得以有意义的所需的数学。

另一方面,量子力学的关键假设之一是,量
\[\label{eq:expectation}
	inner{\varphi}{A\varphi}\]
表示了在态\(\varphi\)下进行可观测量\(A\)的测量的结果期望值,且
\[\label{eq:variance}
	\Delta_\varphi A\defeq\sqrt{\inner{\varphi}{A^2\varphi}-inner{\varphi}{A\varphi}^2}\]
表示了在态\(\varphi\)下进行可观测量\(A\)的测量的不确定度(假定波函数\(\varphi\)归一化到一了)。
并不是\hs\(\cH\)中的任一元素都能计算\nref{期望值}{eq:expectation},只有那些也属于\(\cD(A)\)的\(\varphi\in\cH\)才行。
类似地,\nref{不确定度}{eq:variance}也不能对\hs\(\cH\)中的全部元素计算,而只能对\(\cD(|A|)\)中的计算。%cite
若我们取那些可计算\nref{期望值}{eq:expectation}、\nref{不确定度}{eq:variance}等物理量的可归一化函数为物理上的状态,那么很明显并非任一可平方归一化的函数(即,\(\cH\)中的元素)都能表示一个物理状态。
如我们将见到的,自然的物理波函数空间是\(\cH\)中的一个子空间\(\Phi\),因为其元素的所有物理量都可计算。
进一步来说,\(\Phi\)也具备\dirac 体系的所有好处。

作为例子,我们考虑谐振子。谐振子的代数包含有位置\(Q\)和动量\(P\)这两个可观测量。
这些可观测量定义为\(\cH\)上的线性算子,并满足\ccr:
\[\label{eq:CCR}
	\cmtt{P}{Q}\defeq PQ-QP=-i\hbar I.\]
众所周知的是,\cref{eq:CCR}意味着\(P\)\longRadicalTranslation{或}{either ... or ...}\(Q\)是一个无界算子。
这意味着\(P\)或\(Q\)不能定义于整个\hs 上——它们实际上定义域特定的稠密子域\(\cD(P)\)、\(\cD(Q)\)上,在其上\(P\)和\(Q\)是自伴的。
因此,表达式\(PQ-QP\)不能定义在整个\hs 上。
此外,由于\(\cD(P)\)、\(\cD(Q)\)在\(P\)、\(Q\)的作用下不再是\longRadicalTranslation{不变}{stable}的,表达式\(PQ-QP\)仅对那些满足\(\varphi\in\cD(Q),\varphi\in\cD(P),P\varphi\in\cD(Q),Q\varphi\in\cD(P)\)的\(\varphi\in\cH\)有定义。
因此,\nref{\ccr}{eq:CCR}并非定义于整个\(\cH\)上,而仅定义于其一个子空间。
回想起\cref{eq:CCR}可推出\hur:
\[\label{eq:HUR}
	\Delta_\varphi P\ \Delta_\varphi Q\geq\frac{\hbar}{2}.\]
现在,若希望\(H,P,Q\)的期望值
\[\inner{\varphi}{A\varphi},\quad A=H,P,Q,\]
\(H,P,Q\)的不确定度
\[\label{eq:uncertainty_HPQ}
	\Delta_\varphi A,\quad A=H,P,Q,\]
和\nref{\hur}{eq:HUR}是良定义的,则可平方归一化波函数\(\varphi\)不但是处于\(\cH\)中,还须处于\(\cD(P),\cD(Q),\cD(H),\cD(|P|),\cD(|Q|),\cD(|H|)\)中。

因此,一个使得物理量\crefrange{eq:CCR}{eq:uncertainty_HPQ}能够计算的\(\cH\)的子域\(\Phi\)是必要的。
显然,\(\Phi\)应在\(P,Q,H\)的作用下\silentRadicalTranslation{不变}{stable}。
看上去\(\Phi\)的最佳候选是\(P,Q,H\)的各阶幂的定义域之交:
\[\label{eq:PQH_power_intersec}
	\Phi=\bigcap_{n=0\\A=P,Q,H}^\infty\cD(A^n).\]
\cref{eq:PQH_power_intersec}的空间是谐振子代数的极大不变子空间。
在\(\Phi\)上,所有的物理量如期望、不确定度都可以计算。
代数关系如\ccr 在\(\Phi\)上都是良定义的。
特别地,\hur 在\(\Phi\)上是良定义的。

谐振子的\hamilton 算子的谱是离散的,且其本征矢是可平方归一化的(实际上,它们是\(\phi\)的元素)。
这意味着,就考虑\(H\)的本征矢的而言,没有超出\hs \(\cH\)的必要。
然而,位置、动量可观测量的谱是连续的,且与实数集\longRadicalTranslation{重合}{coincide}。
依照假定\cref{eq:dirac_formalism_feature_1},我们为\(P\)的(连续)谱中的每一元素\(p\)向量\(\ket p\):
\[P\ket p=p\ket p,\quad -\infty<p<+\infty.\]
根据\cref{eq:dirac_formalism_feature_2},一个波函数可用这些本征右矢展开:
\[\label{eq:spectrum_decomposition_over_p}
	\varphi=\infty_{-\infty}^{+\infty}\dd p\ket p\braket{p|\varphi}.\]
显然,右矢\(\ket p\)并不在\hs 中——需要一个更大的线性空间来容纳它们。
\longRadicalTranslation{为此那些\(\ket p\)获得了新的含义,即\(\ket p\in\Phi^\times\),其中\(\Phi^\times\)表示的是空间\(\Phi\)上的反线性泛函所构成的集合。}{It happens that those \(\ket p\) acquire meaning as antilinear functionals
over the space \(\Phi\). That is, \(\ket p\in\Phi^\times\), where \(\Phi^\times\) represents the set of antilinear functionals
over the space \(\Phi\).}
类似的考虑对位置算子\(Q\)也成立:
\[Q\ket x=x\ket x,\quad \ket x\in\Phi^\times,\quad -\infty<x<+\infty.\]
\[\label{eq:spectrum_decomposition_over_x}
	\varphi=\int_{-\infty}^{+\infty}\dd x\ket x\braket{x|\varphi},\quad\varphi\in\Phi.\]
通过这种方式,谐振子的\gelfandtriplet
\[\Phi\subset\cH\subset\Phi^\times\]
就以一种自然的方式出现了。
\hs\(\cH\)是来自于波函数须可平方归一化的要求。
子空间\(\Phi\)是物理波函数的集合,即:可求任意期望、不确定度、对易子的波函数。
对偶空间\(\Phi^\times\)则包含了关联于\longRadicalTranslation{可观测量}{the observables of the algebra}%{原文为「」。疑似应为\bracketedText{可观测量的代数},但}
连续谱的右矢。
这些右矢定义为空间\(\Phi\)上的泛函,并可用于展开任意\(\varphi\in\Phi\),如\cref{eq:spectrum_decomposition_over_p}或\cref{eq:spectrum_decomposition_over_x}。

这些思想将在\cref{chap:3}中详述,在那里我们将构造谐振子的\rhs。
谐振子将以一种不同于量子力学教材的视角被研讨。
谐振子的标准处理手段是从算子代数的(位置)\schrodinger 实现出发,也就是说,将\(Q,P,H\)的众所周知的微分表达式视为是可靠的。
从这些表达式中可推出如\ccr 这样的结果。
亦可推出\hamilton 算子的本征值是可数的,而相应的本征向量由\hermite 多项式给出。
\longRadicalTranslation{也用到了}{are also assumed}\dirac 体系的前述\longRadicalTranslation{预设}{prescription},尽管未提及的是,\hs 的数学并不能与之\longRadicalTranslation{实现}{incorporate}。
在本文中,我们将不会把谐振子代数的位置实现视为是可靠的,而是从代数的假设中推出这一实现。
我们将仅是假定一些\(P,Q,H\)所满足的代数关系,亦即\ccr
\[\cmtt{P}{Q}:=PQ-QP=-i\hbar I,\]
以及用\(P,Q\)写出的\(H\)的表达式
\[H=\frac{1}{2\mu}P^2+\frac{\mu\omega^2}{2}Q^2.\]
我们将追加一个必要的假定,即能量算子至少存在一个本征向量\(\phi_0\):
\[H\phi_0=\frac{\hbar\omega}{2}\phi_0.\]
从这一代数上的出发点,我们可先推出\(H\)有可数个本征值\(\hbar\omega(n+1/2),\ n=0,1,2,\dots\)对应于某些本征向量\(\phi_0\)。
\(\phi_n\)所\longRadicalTranslation{张成}{span}的线性空间将被称为\(\Psi\)。
这一线性空间将配备以两种不同的拓扑:通常的\hs 拓扑,它可从\(\Psi\)中生成\(\cH\);以及另一更强的\longRadicalTranslation{核型}{nuclear}拓扑,它从\(\Psi\)生成了\(\Phi\)。
这个\nuclear 拓扑将使得该代数中的元素都是连续算子。
一旦构造出了\(\Phi\),我们就将构造出\(\Phi^\times\)从而构造出谐振子的\rhs:
\[\label{eq:gelfand_triplet_2}
	\Phi\subset\cH\subset\Phi^\times.\]
本征右矢\(\ket p,\ket x\)将成为\(\Phi\)上的连续反线性泛函,也就是说它们将是\(\Phi^\times\)的元素。
本征右矢方程\(Q\ket x=x\ket x,\ P\ket p=p\ket p\)则\longRadicalTranslation{对应于}{find their mathematical setting as}\(\Phi\)上的泛函方程。
而后会给出\gelfand-\maurin 定理的表述——它将保证位置、动量算子的一组完备广义本征向量集合的存在性,
并展示该定理从数学上证明了那启发性的\dirac 基向量展开\cref{eq:spectrum_decomposition_over_p}和\cref{eq:spectrum_decomposition_over_x}。
我们将推导谐振子的\schrodinger 表示。
在这一表示中,将得到\(P,Q,H\)的基于微分算子的标准表达式。
也将用函数与分布的空间来实现\nref{\RHS}{eq:gelfand_triplet_2}的位置。
空间\(\Phi\)将实现为\schwartz 空间\(\cS(\R)\),而\(\Phi^\times\)将由缓增分布空间\(\cS(\R)^\times\)实现。
因此\nref{\RHS}{eq:gelfand_triplet_2}的位置实现写作
\[\cS(\R)\subset L^2(\R)\subset\cS(\R)^\times.\]
\(H\)的本征向量\(\phi_n\)将由\hermite 多项式实现。

因此,我们将为谐振子的物理所需的操作给出一个适当的数学框架,并关注\bracketedText{谐振子算子代数\schrodinger 实现何以可被单独讨论}的问题。
重要的是,这种本文\emph{\longRadicalTranslation{作为特例}{ad hoc}}引入的实现可从\RHS 框架中的适当代数假定中推出。

\subsection{方势垒的一个\rhs}

量子力学的基本方程是\schrodinger 方程。
因此,说明\RHS 包含有量子力学所需的数学方法就等同于是说明\schrodinger 方程的解的自然框架是\RHS 。
为说明这一点,我们将使用方势垒的例子。%cite

时相关\schrodinger 方程写作
\[\label{eq:t_dep_schrodinger}
	i\hbar\pdo{t}\varphi(t)=H\varphi(t),\]
其中\(H\)表示\hamilton 算子,而\(\varphi(t)\)表示波函数\(\varphi\)中时间\(t\)的值。
\dirac 体系以下面的方式来形式上求解这一方程:对于每个\hamilton 算子的谱\(\sp(H)\)中的能量值\(E\),存在一右矢\(\ket E\)为\(H\)的一个本征向量:
\[\label{eq:eigen_H}
	H\ket E=E\ket E,\quad E\in\sp(H).\]
这些本征右矢形成了一组可展开任意波函数\(\varphi\)如下的完备基:
\[\varphi=\int\dd E\ket E\braket{E|\varphi}\defaseq\int\dd E\varphi(E)\ket E.\]
从\cref{eq:t_dep_schrodinger}所得的时相关解可通过对从\cref{eq:eigen_H}所得的时不相关解进行\fourier 变换来得到:
\[\varphi(t)=\int\dd Ee^{-iEt/\hbar}\varphi(E).\]
若\hamilton 算子的谱有连续的部分,且若能量\(E\)属于该谱的这一连续部分,则对应的作为\cref{eq:eigen_H}的解的本征右矢\(\ket E\)就不是平方可积的,也就是说\(\ket E\)不是\hs 的元素。
如谐振子的情况那样,\hs 并不能处理这些不可归一化的右矢,而\RHS 框架可以。

\RHS 框架的主要短板是它未提供构造\(\Phi,\Phi^\times\)的方法。
核型谱定理的一般表述仅仅是保证了本征右矢\(\ket E\)的存在性,且它假定空间\(\Phi,\Phi^\times\)已事先给出。%cite
因此,我们需要一个为\schrodinger\hamilton 算子构造\RHS 的系统性步骤。
本文的第四章将提供这一系统性步骤。%cite
为使其明晰,我们将在方势垒中阐释这一步骤,尽管同样的方法可适用于更为广泛的一类势能。

构造方势垒的\RHS 的步骤如下。
首先,在径向位置表象中写下时无关\schrodinger 方程:
\[\label{eq:radial_schrodinger}
	\braket{r|H|E}\equiv h\braket{r|E}=E\braket{r|E},\]
其中\(h\)是下面的\schrodinger 微分算子:
\[h\defaseq-\frac{\hbar^2}{2m}\frac{\dd^2}{\dd r^2}+V(r),\]
而
\[V(r)=\begin{cases}
	0&0<r<a\\
	V_0&a<r<b\\
	0&b<r<\infty\\
\end{cases}\]
是方型垒势。
通过对时无关\schrodinger 方程\cref{eq:radial_schrodinger}应用\humanNameTranslation{斯图姆}{Sturm}-\humanNameTranslation{刘维尔}{Liouville}理论(\weyl 理论),可得到一个定义域\(\cD(H)\)使得微分算子\(h\)在其上是自伴的。%cite
