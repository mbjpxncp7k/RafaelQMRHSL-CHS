\chapter{导言}

本章将概述本论文的内容。这些内容主要是关于\dirac 右矢、\lippmann-\schwinger 右矢、\gamow 向量的性质。

% quote

本文是关于\dirac 右矢、\lippmann-\schwinger 右矢、\gamow 向量在\rhs 语言中的表述。
\dirac 右矢是关联于一个可观测量的谱点的态向量。
\lippmann-\schwinger 右矢是与散射理论有关的\hamiltonian 的本征右矢。
它们对应于\longRadicalTranslation{等能}{monoenergetic}“入”散射态和“出”散射态。
\gamow 向量是表示共振的态向量的右矢。
我们的主要目标是展现:\rhs 是对刻画这些右矢而言最合适的框架。
我们将用例子来说明这一点,而不是完全停留于抽象的讨论。主要用到的两个例子是谐振子和方势垒。

本文未讨论任何实验数据,而是关注于解释和理解这些数据的方法、思想和原则。
我们将用配备各种边界条件的\schrodinger 方程来作为提供数据的模型。
\schrodinger 方程上的各种边界条件将给出\dirac 右矢、\lippmann-\schwinger 右矢、\gamow 向量。
这样的模型涉及了一种理想化,但这可能仍是理解这些态向量为何物的最佳方式。

须注意,\RHS\emph{不是}对量子力学的一种阐释,而是一种表述如\dirac 右矢、\lippmann-\schwinger 右矢、\gamow 向量这样的启发性的物理概念的更自然、精确、逻辑化的语言。

\section{\rhs 简史}

在1920年代末,\dirac 引入了量子力学的一种新数学模型,它是基于无穷维复内积向量空间上线性算子所构的一种\longRadicalTranslation{唯一的光滑而精致}{uniquely smooth and elegant}的抽象代数。%cite
\dirac 的\terminology{左矢}、\terminology{右矢}(源于内积的括号记法)的抽象代数模型在之后的年头里已证明是有着巨大的启发价值——尤其是在处理具有连续谱的\hamilton 算子时。
然而,在寻找一种可用于实际数值计算的线性代数时则出现了严重的困难。

\introduceAbbreviation{\hs}{HS}是为量子力学提出的第一种数学\shortRadicalTranslation{形式化}{idealization}模型
然而,如\neumann 在他的书中解释的那样,\HS 理论和\dirac 体系是两回事。尽管有在\hs 中实现\dirac 模型的尝试,仍有不少严重问题使得这体系无法给出左矢、右矢或\deltafunc ,或是无法给\dirac 基向量展开以一个数学含义——而这种展开在涉及连续谱的量子力学的物理表述中都是及其重要的。
当然,\dirac 在他的文字报告中表示:\quoteText{我们目前所用的左矢和右矢构成了一个比\hs 更一般的空间。}。%cite

在1940年代末,%cite
为\deltafunc 给出了精确的含义,它定义为一种测试函数空间上的线性泛函。
这导向了泛函分析的一个新分支的发展:分布理论。%cite

大约是与此同时,\neumann 发表了\hs 的由一个自伴算子诱导的直积分分解理论(也对更一般的情况成立)。%cite
这种谱理论更接近于经典\fourier 分析,它代表了对更早的\neumann 谱理论的一项改进。%cite

%cite
则总是认为\neumann 谱理论并非无穷维向量空间上线性算子理论的完整故事。
获益于分布理论的发展,他与其学派引入了\introduceAbbreviation{\rhs}{RHS}。
从这\rhs 和\neumann 的直积分分解出发,他们得以证明所谓核型谱定理(也称为\gelfand-\maurin 定理)。%cite
该定理为算子的谱性质提供了更为彻底的信息,并在相同立足点上处理了连续谱与离散谱。

作为\dirac 体系的一面,可观测量代数中元素的连续性在1960年代初得到了讨论。%cite
若可观测量代数中的两个算子满足\canonical(\heisenberg)对易关系,则它们中至少有一个不能在\hs 拓扑下是连续的(即,有界的)。
%cite
证明了\hs 中有这样一些子定义域,其上可配备使得那些算子连续的拓扑。而这些子定义域中最大的就是\schwartz 空间。

在1960年代末,一些物理学家独立地意识到\RHS 为\dirac 体系的所有方面都提供了一个严格的数学表述。%cite
特别地,核型谱定理在数学理论中重新表述了\dirac 基向量展开和\dirac 左、右矢。
之后,其他作者也得到了相同结论。%cite
如今,\RHS 已成为教科书中的内容。%cite

在过去一些年中,\RHS 作为散射与衰变理论的自然的数学语言而出现。%cite
\RHS 也被证明在其他理论物理领域中很有用处,如混沌映射的广义谱分解的构造。%cite
实际上,为以一种\longRadicalTranslation{自恰}{consistent}的方式来处理散射与衰变,\RHS 似乎是最为人所知的语言。
这就是我们为何在此使用它。

\schrodinger 方程是统御量子系统行为的动力学方程。
于是任何试图证明\RHS 包含量子力学所需的数学方法的尝试,都应展现\RHS 是\schrodinger 方程\shortRadicalTranslation{之解}{solutions of}的自然框架。
本论文的目的在于得到作为不同边界条件下\schrodinger 方程的解的\dirac、\lippmann-\schwinger、\gamow 右矢,
并展示这些解是落在\RHS 中而非仅仅是在\HS 中。%cite

最终,本文的这些结论将允许我们描绘出一个非常重要的结论:\RHS 是处理散射与衰变的自然语言。

\section{谐振子}